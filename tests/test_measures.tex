\section{Spannungsstabilisierung}

\subsection{Einführung}

\begin{sagesilent}
    U_in0 = 10
    I_out_max = 0.05
    I_ZD = 0.001

    U_ZD = 5.1
    U_ZD_max = 5.4
    U_BE = 0.66
\end{sagesilent}

Angabe: $U_{in0} = \SIvar{U_in0}{\volt}$, $I_{out_{max}} = \SIvar{I_out_max}{\ampere}$, $I_{ZD} = \SIvar{I_ZD}{\ampere}$

Datenblatt: $U_{ZD} = \SIvar{U_ZD}{\volt}$, $U_{ZD_{max}} = \SIvar{U_ZD_max}{\volt}$, $U_{BE} = \SIvar{U_BE}{\volt}$

\subsection{Schaltung}

\begin{figure}[H]
    \centering
    \begin{circuitikz}
        \coordinate (in+) at (0,0);
        \coordinate (in-) at (0,-5);

        \draw
        (in+) to[short,o-] ++(2,0) coordinate (aux1)
        to[R=$R_{V}$,*-*] ++(0,-2) coordinate (aux2)
        to[zD=$ZD$,name=ZD,i>^,invert,*-*] (aux2 |- in-)

        (aux2) ++(2,0) node[npn] (T) {$T$}
        (T.E) coordinate (out+)

        (aux2) to[short,name=I_B,i] (T.B)
        (aux1) -- (aux1 -| T.C) -- (T.C)

        (out+) to[R=$R_{L}$,name=R_L,i>^,o-o] (out+ |- in-) coordinate (out-)

        (in-) to[short,o-] (out-)

        (in+) to[open,name=inV,v] (in-)
        ($(out+) + (1,0)$) to[open,name=outV,v^] ($(out-) + (1,0)$)
        ;

        \voltage{inV}{$U_{in0}$}
        \current{ZD}{$I_{ZD}$}
        \current{I_B}{$I_{B}$}
        \current{R_L}{$I_{out}$}
        \voltage{outV}{$U_{out}$}
    \end{circuitikz}
\end{figure}

\subsection{Berechnungen}

\begin{sagesilent}
    R_V = (U_in0 - U_ZD) / I_ZD
    R_L_min = (U_ZD_max - U_BE) / I_out_max
\end{sagesilent}

\begin{equation*}
    R_{V} = \frac{U_{in0} - U_{ZD}}{I_{ZD} + I_{B}} \tag*{$I_{B} = \quad \ll$ \quad \textit{(vernachlässigbar)}}
\end{equation*}

\begin{align*}
    R_{V} &= \frac{U_{in0} - U_{ZD}}{I_{ZD}} \\
    R_{V} &= \frac{\var{U_in0} - \var{U_ZD}}{\var{I_ZD}} \\
    R_{V} &= \SIvar{R_V}{\ohm}
\end{align*}

\begin{align*}
    R_{L_{min}} &= \frac{U_{ZD_{max}} - U_{BE}}{I_{out_{max}}} \\
    R_{L_{min}} &= \frac{\var{U_ZD_max} - \var{U_BE}}{\var{I_out_max}} \\
    R_{L_{min}} &= \SIvar{R_L_min}{\ohm}
\end{align*}

\subsection{Simulation}

\begin{figure}[H]
    \centering
    \includesvg[width=0.8\textwidth]{tests/simulations/export/test_measures.svg}
    \caption{Ausgangskennlinie \textbf{$U_{out} = f(R_L)$} der Simulation. \textit{(als SVG eingebettet)}}
\end{figure}

\begin{sagesilent}
    simulation_data = LTSpice.plot_data("tests/simulations/export/test_measures.txt")
    simulation_data_ohm_volt = simulation_data["V(out)"]

    simulation_plot = list_plot(
        simulation_data_ohm_volt,
        axes_labels=["$R_L$ in $\Omega$", "$U_{out}$ in V"],
        color='#148873',
        figsize=[5.5,2],
        plotjoined=True,
        frame=True
    )
\end{sagesilent}

\begin{figure}[H]
    \centering
    \sageplot{simulation_plot}
    \caption{Ausgangskennlinie \textbf{$U_{out} = f(R_L)$} der Simulation. \textit{(importiert aus ltspice)}}
\end{figure}

\subsection{Auswertung}

\begin{sagesilent}
    measured_data = Measures(
        [
            "$R_L$ in $\Omega$",
            "$U_{out}$ in V", 
            "$I_{out}$ in A"
        ], [
            [100, 3.96, 38.2],
            [120, 3.96, 32.0],
            [130, 3.93, 29.5],
            [140, 3.94, 27.4],
            [150, 3.95, 25.7],
            [160, 3.94, 24.1],
            [170, 3.92, 22.6],
            [200, 3.91, 19.2],
            [250, 3.95, 15.6],
            [300, 3.98, 13.2],
            [400, 4.05, 10.0],
            [500, 4.10, 8.20],
            [600, 4.17, 6.95],
            [700, 4.22, 6.00],
            [800, 4.23, 5.30],
            [1000, 4.29, 4.30],
            [2000, 4.30, 2.16],
    ], ndigits=[0, 2, 1])

    index_RL = 0
    index_Ua = 1
    index_Ia = 2
\end{sagesilent}

\subsubsection{Messdaten}

\begin{table}[H]
    \centering
    \renewcommand{\arraystretch}{1.2}
    \sage{measured_data.table}
\end{table}

\subsubsection{Grafische Darstellung}

\begin{sagesilent}
    measured_data_plot = list_plot(
        measured_data.plot_data(index_RL, index_Ua),
        axes_labels=["$R_L$ in $\Omega$", "$U_{out}$ in V"],
        color='purple',
        figsize=[5.5,2]
    )

    measured_data_plot += line(
        measured_data.g1d_smooth(index_RL, index_Ua, sigma=0.75, connect=True),
        linestyle="--",
        color='blue'
    )

    measured_data_plot += line(
        [(100, 4), (100, 3.9)],
        linestyle=":",
        color="red"
    )

    measured_data_plot += line(
        [(200, 4), (200, 3.9)],
        linestyle=":",
        color="red"
    )
\end{sagesilent}

\begin{figure}[H]
    \centering
    \sageplot{measured_data_plot}
    \caption{Ausgangskennlinie \textbf{$U_{out} = f(R_L)$} mit Messdaten}
\end{figure}