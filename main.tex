\documentclass[a4paper]{hitec}
\settextfraction{0.95}      % reduce left margin

\usepackage{styles/main}
\usepackage{styles/custom}

\labtitle{}

\author{}
\company{HTBLA Weiz}
\date{}

\begin{document}
\begin{sagesilent}
    load_attach_path("../helpers/")
    load("norm.sage")
    load("measures.sage")
    load("bode.sage")
\end{sagesilent}

\maketitletoc
\clearpage

\section{Einführung}

\begin{sagesilent}
    R = 95.5
\end{sagesilent}

Angaben: $R = \sage{norm(R)}\Omega$

\section{Schaltung}

\begin{figure}[H]
    \centering
    \begin{circuitikz}
        \draw
          (0,0)     coordinate (in+)
        ++(4,0)     coordinate (out+)
          (0,-3)    coordinate (in-)
        ++(4,0)     coordinate (out-)

        (in+) to[R=$R$,i=$I$,o-o] (out+)
        (in-) to[short,o-o] (out-)
        
        (in+)   to[open, v=$U_e$] (in-)
        (out+)  to[open, v^=$U_a$] (out-)
        ;
    \end{circuitikz}
\end{figure}

\section{Berechnungen}

\begin{equation*}
    \frac{U}{R \cdot I}
\end{equation*}

\begin{sagesilent}
    I = 0.005
    U = R * I
\end{sagesilent}

\begin{align*}
    U &= R \cdot I \\
    U &= \sage{norm(R)} \cdot \sage{norm(I)} \\
    U &= \sage{norm(U)}V
\end{align*}

\section{Simulation}

\section{Auswertung}

\begin{sagesilent}
    measures = Measures(
        [
            '$R$ in $\Omega$',
            '$I$ in $A$', 
            '$U_a$ in $V$'
        ], [
            [norm(R), '0.005', '0.4775'],
            [norm(R), '0.010', '0.4775']
    ])

    List_R = measures.columns[0]
    List_I = measures.columns[1]
    List_Ua = measures.columns[2]
\end{sagesilent}

\subsection{Messwerte}

\begin{center}
    \renewcommand{\arraystretch}{1.2}
    \sage{measures.table}
\end{center}

\subsection{Grafische Darstellung}

\begin{sagesilent}
    f = list_plot(
        list(zip(List_I, List_Ua)),
        color='blue',
        plotjoined=True,
        figsize=[5,2],
        axes_labels=['$I$ in $A$', '$U_a$ in $V$'],
        axes_labels_size=1
    )
\end{sagesilent}

\begin{figure}[H]
    \centering
    \sageplot{f}
    \caption{Ausgangskennlinie mit gemessene Werte}
\end{figure}

\clearpage

\subsection{Verwendete Geräte}

\medskip

\begin{devicelist}
    \devicecat{}{2}{
        \device{}{}
        \device{}{}
    }
\end{devicelist}

\vfill

\IncludeHistoryTimeline

\end{document}